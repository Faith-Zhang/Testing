\documentclass{article}
\usepackage[utf8]{inputenc}

\title{Testing}

\date{August 2019}
\author{Faith}
\begin{document}

\maketitle
If a change a lot of the stuff, what's gonna happen?
testing \\
testing\\
And I deleted this line
\\
\\
check\\


Let's say I add a row here
check\\
check\\
Let's remove all these lines and see what will happen to my desktop file!
\\
just for fun
\\

If I edit here, can I expect changes when I open it in the web?? I hope so.


=======
This is really weird stuff!


Now it works, but still a bit weird!

I see I see I see 
\end{document}
